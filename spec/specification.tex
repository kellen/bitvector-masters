\documentclass[a4paper,11pt]{article}
\usepackage[british]{babel}
\usepackage{color}
\usepackage{fullpage}
\usepackage{hyperref} % should always be the last package

\newcommand{\todo}[1]{{\color{red}(To do: #1)}} % comment away to OMIT to-do items
%\newcommand{\todo}[1]{}                        % comment away to SHOW to-do items

\newcommand{\AllDifferent}{\Constraint{AllDifferent}}
\newcommand{\Constraint}[1]{\textsc{#1}}
\newcommand{\Comet}{\textit{Comet}}
\newcommand{\Gecode}{\textit{Gecode}}
\newcommand{\Set}[1]{\left\{#1\right\}}

\title{Implementing Bit-Vector Variables,\\ with an Application to the
  Generation of S-Boxes}

\author{Specification of a Master Thesis in Computer Science \\
  at the Department of Information Technology at Uppsala University \\
  by
  \href{mailto:Kellen.Dye.0894@student.uu.se}{Kellen Dye}
  (19810421-1217)
}

\date{\today}

\begin{document}

\maketitle

\noindent
This master thesis project (D-level, 20 weeks, 30 credits) will be
done at the ASTRA research group
(\url{http://www.it.uu.se/research/group/astra}) at Uppsala
University.  It will be supervised by Dr Jean-No\"el Monette and reviewed
by Prof.\ Pierre Flener.

\section*{Background: Constraint Programming}

\textbf{Constraint programming} (CP) \cite{constraintshandbook} is a
set of techniques to find values for some given decision variables
within their given domains, so that some given constraints on these
variables are satisfied and possibly that some cost (or benefit)
expression on these variables is minimised (or maximised).

The core concept of CP is the notion of \textbf{constraint}, which is
a declarative encapsulation of specialised algorithms, discussed
below.  Beside simple comparison ($<$, $\leq$, $=$, $\neq$, $\geq$,
$>$) and arithmetic (linear equation, disequation, and inequation)
constraints, a large number of combinatorial constraints have been
identified and equipped with algorithms. Those constraints capture
recurring problem substructures, such as
$\AllDifferent(\Set{x_1,\dots,x_n})$, which means that the $x_i$
decision variables take pairwise distinct values.

CP solves such combinatorial problems by \textbf{systematic search}
with backtracking, and performs at every node of the search tree a
particular kind of inference called \textbf{propagation}.  For each
constraint, a \textbf{propagation algorithm} narrows the domains of
its decision variables by eliminating impossible values according to
some desired level of \textbf{consistency}.  Propagation at a node of
the search tree then amounts to computing the greatest simultaneous
fixpoint of the propagators for all the constraints.

Classically, decision variables take their values in a subset of the
integers. However other domains have been proposed, e.g., floating
point numbers, sets, and graphs. Recent work has been performed on the
domain of \textbf{bit-vector} variables in CP. Such variables take
their values in a subset of all possible vectors of bits (of a given
length). Such a domain has applications in many areas of computer
science, e.g., in cryptography or program verification.

\section*{Task Description}

The goal of this thesis is to provide an implementation of the
bit-vector domain in CP and to apply it on relevant problems.

More precisely, the work can be roughly divided into the following
goals:

\begin{itemize}
\item Implement the bit-vector variables as described in~\cite{bitvectors}. 
\item Design and implement propagation algorithms, for both primitive
  and global constraints applied on bit-vectors.
\item Use the bit-vector variables and constraints to design
  cryptographic S-boxes following the approach outlined in~\cite{sboxes}.
\end{itemize}

\section*{Procedure}

\begin{itemize}
\item Meetings will be held by appointment or on request by the
  student, supervisor, or reviewer. 
\item All implementation will be done in \Gecode, a free software
  (available at \url{http://www.gecode.org/}), under the terms of the
  MIT license.
\item The thesis will be written in English using \LaTeX.
\item A paper will be written with the supervisor and reviewer if the
  results warrant it.  It will be submitted to some relevant CP event.
\item All documents and software produced during this work will be
  archived at ASTRA.
\end{itemize}


\section*{Time Plan}

The equivalent of 20 full-time weeks will be spent on
this work, beginning in week 4 of 2014.
We aim at completion by late June 2014, upon a public presentation of
the results and revision of the thesis according to the comments by
the audience and the reviewer.  The time will be spent according to
the following time plan:
\begin{enumerate}
\item Background study on \Gecode{} and bit-vector solvers (2~weeks).
\item Implementation of the bit-vector variables and primitive
  constraints (3~weeks).
\item Experimental comparison of the implementation with existing
  approaches (3~weeks)
\item Design and implementation of a CP model for the generation of
  S-boxes (3~weeks).
\item Design and implementation of global constraints that improve the CP model for S-boxes (3~weeks).
\item Experimental evaluation of the CP model for S-boxes (2~weeks).
\item Writing of the thesis (4~weeks).
\end{enumerate}

\bibliographystyle{abbrv}
\bibliography{specification}

\end{document}
