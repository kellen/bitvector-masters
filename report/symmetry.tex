Ramamoorthy et al.\ define several types of symmetry: 
bit-inversion symmetry in which the value of an S-box $S(x)$ is replaced with its inverse, $\lnot S(x)$,
rotational symmetry for a $180^{\circ}$ rotation,
row symmetry in which certain rows are exchanged,
column symmetry in which certain columns are exchanged,
and
diagonal symmetry in which blocks of values in the S-box are exchanged diagonally~\cite{sboxsymmetry}.
We additionally identify a reflective symmetry over the $x$ axis and a reflective symmetry over the $y$ axis.

\subsection{Reflection over the $x$-axis}

Reflecting an S-box $S$ over the $x$-axis produces a new S-box, $S'$ in which the values in row 1 are exchanged for those in row 4 and the values in row 2 are exchanged for those in row 3, and vice-versa.
Reflecting a $6 \times 4$ S-box $S$, for any input $x$ to $S$, the value at $S(x)$ will be moved to 
input $x \oplus 100001$. For an \nbym{} S-box this is 
$x \oplus (2^{n-1} + 2^0)$.

For this to be a symmetry, the constraints applied to $S$ should not be violated for $S'$, and conversely.
The \sc{DES} criterion S-1 is already fulfilled by the choice of variables.

\subsubsection{Non-linearity criterion S-2}

The S-2 criterion requires that no output bit should be "too close" to a linear function of the input bits.
This measure of linearity is captured in the count $\N$ and the score $\sigma$.

Recall that the linear combination of two bit-vectors $x$ and $y$ of equal length is:
\[\lin(x,y) = (x_0 \land y_0) \oplus (x_1 \land y_1) \oplus \ldots \oplus (x_{n-1} \land y_{n-1})\]
If it is known that a bit $k$ is $0$ in $x$, then $\lin(x + 2^{k},y) = \lin(x) \oplus y_k$.

Recall that the count of the S-box is defined as:
\[\N = \Cardinality{\set{x\,\middle|\,0\leq x < 2^n,\,\lin(x,\alpha) = \lin(S(x),\beta)}}\] 

It is possible to rewrite $\N$ as a column-based sum (assuming four rows):
\[
    \N = \sum_{i=0}^{2^{n-2}-1} \left(
\begin{aligned}
    &1 \oplus \lin(2i,\alpha) \oplus \lin(S(2i),\beta) + \\
    &1 \oplus \lin(2i + 1,\alpha) \oplus \lin(S(2i + 1),\beta) + \\
    &1 \oplus \lin(2i + 2^{n-1},\alpha) \oplus \lin(S(2i + 2^{n-1}),\beta) + \\
    &1 \oplus \lin(2i + 2^{n-1} + 1,\alpha) \oplus \lin(S(2i + 2^{n-1} + 1),\beta)
\end{aligned}
    \right)
\]
The top line of the sum corresponds to the first row of the S-box, the second line to the second row, and so on.
Due to the range of $i$, the $0$-th and $n-1$-st bits of $2 \cdot i$ will always be zero
and because of this, terms like 
$\lin(2i + 2^{n-1} + 1)$ can be reduced:
\[
    \N = \sum_{i=0}^{2^{n-2}-1} \left(
\begin{aligned}
    &1 \oplus \lin(2i,\alpha) \oplus \lin(S(2i),\beta) + \\
    &1 \oplus \lin(2i,\alpha) \oplus \alpha_0 \oplus \lin(S(2i + 1),\beta) + \\
    &1 \oplus \lin(2i,\alpha) \oplus \alpha_{n-1} \oplus \lin(S(2i + 2^{n-1}),\beta) + \\
    &1 \oplus \lin(2i,\alpha) \oplus \alpha_0 \oplus \alpha_{n-1} \oplus \lin(S(2i + 2^{n-1} + 1),\beta)
\end{aligned}
    \right)
\]

Reflecting over the $x$ axis alters the values in each column. 
The corresponding $\Nprime$ for the reflection of the S-box over the $x$ axis involves swapping the values $S(2i)$ and $S(2i + 2^{n-1} + 1)$ and the values $S(2i+1)$ and $S(2i + 2^{n-1})$ to get the sum: 
\[
    \Nprime = \sum_{i=0}^{2^{n-2}-1} \left(
\begin{aligned}
    &1 \oplus \lin(2i,\alpha) \oplus \lin(S(2i + 2^{n-1} + 1),\beta) + \\
    &1 \oplus \lin(2i,\alpha) \oplus \alpha_0 \oplus \lin(S(2i + 2^{n-1}),\beta) + \\
    &1 \oplus \lin(2i,\alpha) \oplus \alpha_{n-1} \oplus \lin(S(2i + 1),\beta) + \\
    &1 \oplus \lin(2i,\alpha) \oplus \alpha_0 \oplus \alpha_{n-1} \oplus \lin(S(2i),\beta)
\end{aligned}
    \right)
\]

When $a_0$ and $a_{n-1}$ are both equal to $0$ or both equal to $1$, observe that $\N = \Nprime$. For $a_0 = a_{n-1} = 1$:
\begin{align*}
    \N &= 
    \sum_{i=0}^{2^{n-2}-1} \left(
\begin{aligned}
    &1 \oplus \lin(2i,\alpha) \oplus \lin(S(2i),\beta) + \\
    &1 \oplus \lin(2i,\alpha) \oplus 1 \oplus \lin(S(2i + 1),\beta) + \\
    &1 \oplus \lin(2i,\alpha) \oplus 1 \oplus \lin(S(2i + 2^{n-1}),\beta) + \\
    &1 \oplus \lin(2i,\alpha) \oplus 1 \oplus 1 \oplus \lin(S(2i + 2^{n-1} + 1),\beta)
\end{aligned}
    \right) \\
    &=
    \sum_{i=0}^{2^{n-2}-1} \left(
\begin{aligned}
    &1 \oplus \lin(2i,\alpha) \oplus \lin(S(2i),\beta) + \\
    &\lin(2i,\alpha) \oplus \lin(S(2i + 1),\beta) + \\
    &\lin(2i,\alpha) \oplus \lin(S(2i + 2^{n-1}),\beta) + \\
    &1 \oplus \lin(2i,\alpha) \oplus \lin(S(2i + 2^{n-1} + 1),\beta)
\end{aligned}
    \right) \\
    \Nprime &= 
    \sum_{i=0}^{2^{n-2}-1} \left(
\begin{aligned}
    &1 \oplus \lin(2i,\alpha) \oplus \lin(S(2i + 2^{n-1} + 1),\beta) + \\
    &1 \oplus \lin(2i,\alpha) \oplus 1 \oplus \lin(S(2i + 2^{n-1}),\beta) + \\
    &1 \oplus \lin(2i,\alpha) \oplus 1 \oplus \lin(S(2i + 1),\beta) + \\
    &1 \oplus \lin(2i,\alpha) \oplus 1 \oplus 1 \oplus \lin(S(2i),\beta)
\end{aligned}
    \right) \\
    &=
    \sum_{i=0}^{2^{n-2}-1} \left(
\begin{aligned}
    &1 \oplus \lin(2i,\alpha) \oplus \lin(S(2i + 2^{n-1} + 1),\beta) + \\
    &\lin(2i,\alpha) \oplus \lin(S(2i + 2^{n-1}),\beta) + \\
    &\lin(2i,\alpha) \oplus \lin(S(2i + 1),\beta) + \\
    &1 \oplus \lin(2i,\alpha) \oplus \lin(S(2i),\beta)
\end{aligned}
    \right) \\
    &= \N
\end{align*}
The same equality holds if $a_0 = a_{n-1} = 0$. 

When $a_0 \neq a_{n-1}$, however, $\Nprime = 2^n - \N$. For $a_0 = 0, a_{n-1} = 1$:
\begin{align*}
    \N &= 
    \sum_{i=0}^{2^{n-2}-1} \left(
\begin{aligned}
    &1 \oplus \lin(2i,\alpha) \oplus \lin(S(2i),\beta) + \\
    &1 \oplus \lin(2i,\alpha) \oplus 0 \oplus \lin(S(2i + 1),\beta) + \\
    &1 \oplus \lin(2i,\alpha) \oplus 1 \oplus \lin(S(2i + 2^{n-1}),\beta) + \\
    &1 \oplus \lin(2i,\alpha) \oplus 0 \oplus 1 \oplus \lin(S(2i + 2^{n-1} + 1),\beta)
\end{aligned}
    \right) \\
    &=
    \sum_{i=0}^{2^{n-2}-1} \left(
\begin{aligned}
    &1 \oplus \lin(2i,\alpha) \oplus \lin(S(2i),\beta) + \\
    &1 \oplus \lin(2i,\alpha) \oplus \lin(S(2i + 1),\beta) + \\
    &\lin(2i,\alpha) \oplus \lin(S(2i + 2^{n-1}),\beta) + \\
    &\lin(2i,\alpha) \oplus \lin(S(2i + 2^{n-1} + 1),\beta)
\end{aligned}
    \right) \\
    \Nprime &= 
    \sum_{i=0}^{2^{n-2}-1} \left(
\begin{aligned}
    &1 \oplus \lin(2i,\alpha) \oplus \lin(S(2i + 2^{n-1} + 1),\beta) + \\
    &1 \oplus \lin(2i,\alpha) \oplus 0 \oplus \lin(S(2i + 2^{n-1}),\beta) + \\
    &1 \oplus \lin(2i,\alpha) \oplus 1 \oplus \lin(S(2i + 1),\beta) + \\
    &1 \oplus \lin(2i,\alpha) \oplus 0 \oplus 1 \oplus \lin(S(2i),\beta)
\end{aligned}
    \right) \\
    &=
    \sum_{i=0}^{2^{n-2}-1} \left(
\begin{aligned}
    &1 \oplus \lin(2i,\alpha) \oplus \lin(S(2i + 2^{n-1} + 1),\beta) + \\
    &1 \oplus \lin(2i,\alpha) \oplus \lin(S(2i + 2^{n-1}),\beta) + \\
    &\lin(2i,\alpha) \oplus \lin(S(2i + 1),\beta) + \\
    &\lin(2i,\alpha) \oplus \lin(S(2i),\beta)
\end{aligned}
    \right)
\end{align*}
Notice that when the term for a row is added to $\N$, its inverse is added to $\Nprime$. 
Since these are Boolean terms and there are four rows and $2^{n-2}$ columns of terms to be summed,
$\N + \Nprime = 2^{n-2} \cdot 2^2 = 2^n$, which then implies that
$\Nprime = 2^n - \N$.
The same equality holds for $\alpha_0 = 1, \alpha_{n-1} = 0$. 

Recall that the score, $\sigma$, of an S-box is defined as
\[\sigma = \max_{\alpha,\beta}\left\{\abs{\N - \frac{2^n}{2}}\right\}\]

Let $A = \abs{\N - \sfrac{2^n}{2}}$. 
The corresponding $A'$ for the reflected S-box has two cases, 
one for $\alpha_0 = \alpha_{n-1}$
and another for $\alpha_0 \neq \alpha_{n-1}$:
\begin{align*}
A' &= \begin{dcases*}
        \abs{\Nprime - \sfrac{2^n}{2}} & if $\alpha_0 = \alpha_{n-1}$ \\
        \abs{\Nprime - \sfrac{2^n}{2}} & if $\alpha_0 \neq \alpha_{n-1}$
    \end{dcases*} \\
   &= \begin{dcases*}
        \abs{\N - \sfrac{2^n}{2}} & if $\alpha_0 = \alpha_{n-1}$ \\
        \abs{2^n - \N - \sfrac{2^n}{2}} & if $\alpha_0 \neq \alpha_{n-1}$ 
    \end{dcases*} \\
   &= \begin{dcases*}
        \abs{\N - \sfrac{2^n}{2}} & if $\alpha_0 = \alpha_{n-1}$ \\
        \abs{\N - \sfrac{2^n}{2}} & if $\alpha_0 \neq \alpha_{n-1}$
    \end{dcases*} \\
   &= \abs{\N - \sfrac{2^n}{2}} \\
   &= A
\end{align*}

So the score for the reflected S-box, $\sigma'$, is the same as $\sigma$:
\begin{align*}
\sigma' = \max_{\alpha,\beta}\left\{A'\right\}
        = \max_{\alpha,\beta}\left\{A\right\}
        = \sigma
\end{align*}
%\begin{align*}
%\sigma' &= \max_{\alpha,\beta}\left\{A'\right\} \\
%        &= \max_{\alpha,\beta}\left\{A\right\} \\
%        &= \sigma
%\end{align*}

Since the reflected S-box has the same score as the original S-box, 
the original S-box fulfills criterion S-2
if and only if 
the reflected S-box
also fulfills criterion S-2.

\subsubsection{Criterion S-3}
Criterion S-3 requires that the values of each row are distinct; the original S-box has this property if and only if the reflected S-box also has this property.

\subsubsection{Criterion S-4}

Criterion S-4 is a constraint on pairs of bit-vector variables, $x$ and $y$, 
whose inputs differ by a single bit,
that is that $\weight(x \oplus y) = 1$.
In the reflected S-box, $S'$, the input
$x$ is moved to $x \oplus (2^{n-1} + 2^0)$ and
the input $y$ is moved to 
$y \oplus (2^{n-1} + 2^0)$.
Because 
$x \oplus (2^{n-1} + 2^0) \oplus y \oplus (2^{n-1} + 2^0) = x \oplus y$,
reflecting $S$ to $S'$ maintains this relationship between $x$ and $y$,
so criterion S-4 is fulfilled in $S$ if and only if it is also fulfilled in $S'$.
%Criterion S-4 is a constraint on pairs of variables whose inputs differ by a single bit. 
%If the single differing bit is one of the middle bits, then the relationship is between two variables belonging to the same row of the S-box, whose relationship will be maintained upon reflection.
%
%If the single differing bit is either the first or the last bit, 
%then the relationship is between variables in the same column of the S-box. 
%Recall that each row is defined by the first and last bits of the input bit-vector, 
%so reflecting the S-box is equivalent to inverting both the first and last bits of input. 
%Due to this property, each input of the S-box will be related to the other members of the column 
%\textit{except} for the position to which it will be reflected (since this is a change of \textit{two} bits).
%The relationships between these S-box values will also therefore be maintained over reflection.

\subsubsection{Criterion S-5}
Criterion S-5 is a constraint on pairs of bit-vector variables whose inputs differ in the "two middle bits exactly." 
Variables which have this property belong to the same row of the S-box, and these middle bits are not 
affected by the reflection, so this constraint is fulfilled by $S$ if and only if it is also fulfilled by $S'$.

\subsubsection{Criterion S-6}
Criterion S-6 is a constraint on pairs of bit-vector variables whose inputs differ in their first two bits,
but are the same in their last two.

For a given input $x$, the inputs related to $x$ are
several variables $y$ such that 
$(x \oplus y) \land (2^{n-1} + 2^{n-2} + 2^1 + 2^0) = (2^{n-1} + 2^{n-2})$.
In the reflected S-box, $S'$, the input $x$ is moved to 
$x \oplus (2^{n-1} + 2^0)$ 
and 
$y$ is moved to 
$y \oplus (2^{n-1} + 2^0)$.
And because 
$x \oplus (2^{n-1} + 2^0) \oplus y \oplus (2^{n-1} + 2^0) = (x \oplus y)$, 
the relationship between these inputs is maintained in $S'$.

\subsubsection{Criterion S-7}
Criterion S-7 places a constraint on sets of pairs of bit-vector variables whose inputs $x$ and $y$ 
differ for some input difference $i$: $x \oplus y = i$. 

Again, because
$x \oplus (2^{n-1} + 2^0) \oplus y \oplus (2^{n-1} + 2^0) = (x \oplus y)$,
this relationship is unchanged for the reflected S-box $S'$,
so criterion S-7 is fulfilled in $S$ if and only if it is also fulfilled in $S'$.

\subsection{Reflection over the $y$-axis}

We have shown that reflection over the $x$ axis is a symmetry, while
Ramamoorthy et al.\ have shown a $180^{\circ}$ degree rotational symmetry.
A reflection over the $y$-axis is equivalent to a reflection over the $x$ axis 
followed by a $180^{\circ}$ rotation.

If an S-box $S$ fulfills criteria S-1--S-7, 
then so does an S-box $S'$ which is reflected over the $x$-axis,
and so does an S-box $S''$ which is $S'$ rotated by $180^{\circ}$, 
which in turn is equivalent to an S-box reflected over the $y$-axis.

\subsection{Symmetry-breaking}
\label{sec:breakreflection}

During search, reflective symmetry over the $x$-axis 
can be broken by constraining the variable 
at the top-left of the S\=/box to be less than or equal to the 
variable at the bottom-left of the S\=/box.
For a \sixbyfour{} S\=/box, these are the variables for
input $0$ and $33$:
$I(S(0)) \leq I(S(33))$.

The reflective symmetry over the $y$-axis can be broken during search
by constraining the variable at the top-left of the
S\=/box to be less than or equal to
the variable at the top-right of the S\=/box.
For a \sixbyfour{} S\=/box, these are the variables for
input $0$ and $30$:
$I(S(0)) \leq I(S(30))$.
