\documentclass{article}
\usepackage{mathtools}
\usepackage{tikz}                   % draw pictures
\usepackage{pgfplots}               % draw plots
%\pgfplotsset{compat=1.7}
\usepackage{pgfplotstable}          % fuck yeah! tables from tab-separated data
\usetikzlibrary{calc,decorations.markings,positioning} % draw bus-width lines
\usepackage[outline]{contour}
\contourlength{0.5pt}
%\tikzset{external/system call={pdflatex \tikzexternalcheckshellescape -halt-on-error -interaction=batchmode -jobname "\image" "\texsource"}}
%\tikzset{external/system call={lualatex \tikzexternalcheckshellescape -halt-on-error -interaction=batchmode -jobname "\image" "\texsource"}}
%\usetikzlibrary{external}           % externalize the tikz pictures for faster compilation
\begin{document}
Cat dog \begingroup\medmuskip=0mu\relax$6 \times 4$\endgroup dog cat

\pgfmathsetmacro{\figscale}{1}
\begin{figure}
    \centering
    \begin{tikzpicture}[thick,scale=\figscale, every node/.style={transform shape}, font=\sffamily]
	\begin{scope}
	%\begin{scope}[decoration={markings,mark=at position 0.5 with {\arrow{>}}}]
        \pgfmathsetmacro{\numscale}{1.5}

		% left horizontal
		\draw (0, 0) -- (2, 0);
		\draw (0, 1) -- (2, 1);
		\draw (0, 2) -- (2, 2);
		\draw (0, 3) -- (2, 3);
		\draw (0, 4) -- (2, 4);
		% right horizontal
		\draw (5, 0) -- (7, 0);
		\draw (5, 1) -- (7, 1);
		\draw (5, 2) -- (7, 2);
		\draw (5, 3) -- (7, 3);
		\draw (5, 4) -- (7, 4);

		% dotted horizontal left
		\draw [densely dotted] (2, 0) -- ++(0.75, 0);
		\draw [densely dotted] (2, 1) -- ++(0.75, 0);
		\draw [densely dotted] (2, 2) -- ++(0.75, 0);
		\draw [densely dotted] (2, 3) -- ++(0.75, 0);
		\draw [densely dotted] (2, 4) -- ++(0.75, 0);
		% dotted horizontal right
		\draw [densely dotted] (5, 0) -- ++(-0.75, 0);
		\draw [densely dotted] (5, 1) -- ++(-0.75, 0);
		\draw [densely dotted] (5, 2) -- ++(-0.75, 0);
		\draw [densely dotted] (5, 3) -- ++(-0.75, 0);
		\draw [densely dotted] (5, 4) -- ++(-0.75, 0);

		% left vertical
		\draw (0, 0) -- (0, 4);
		\draw (1, 0) -- (1, 4);
		\draw (2, 0) -- (2, 4);
		% right vertical
		\draw (5, 0) -- (5, 4);
		\draw (6, 0) -- (6, 4);
		\draw (7, 0) -- (7, 4);

		% top left zig-zag 
		\draw [red] (0.5,3.5) 
		-- ++(0, -1)
		-- ++(1,1)
		-- ++(0, -1);
		% bottom left zig zag
		\draw [red] (0.5,1.5) 
		-- ++(0, -1)
		-- ++(1,1)
		-- ++(0, -1);

		% top left continuation
		\draw [red,densely dotted] (1.5,2.5) 
		-- ++(1,1) 
		-- ++(0,-1)
		-- ++(0.5,0.5);
		% bottom left continuation
		\draw [red,densely dotted] (1.5,0.5) 
		-- ++(1,1) 
		-- ++(0,-1)
		-- ++(0.5,0.5);

		% top right continuation
		\draw [red,densely dotted] (4.5,3.5) 
		-- ++(0,-1)
		-- ++(1,1);
		% bottom right continuation
		\draw [red,densely dotted] (4.5,1.5) 
		-- ++(0,-1)
		-- ++(1,1);

		% top right zig zag
		\draw [red] (5.5,3.5) 
		-- ++(0, -1)
		-- ++(1,1)
		-- ++(0, -1);
		% bottom right zig zag
		\draw [red] (5.5,1.5) 
		-- ++(0, -1)
		-- ++(1,1)
		-- ++(0, -1);

		% connecting top and bottom
		% angle: phi = arctan(1/6) = 9.46 degrees
		%\draw [red] (0.5,1.5) -- (6.5, 2.5);
		\draw [red,densely dotted] (0.5,1.5) -- (6.5, 2.5);
		\draw [red] (0.5,1.5) -- ++(9.46:1.75);
		\draw [red] (6.5,2.5) -- ++(189.46:1.75);

		% numbers...
		\node [anchor=center,scale=\numscale] at (0.5,3.5) {\contour{white}{0}};
		\node [anchor=center,scale=\numscale] at (0.5,2.5) {\contour{white}{1}};
		\node [anchor=center,scale=\numscale] at (1.5,3.5) {\contour{white}{2}};
		\node [anchor=center,scale=\numscale] at (1.5,2.5) {\contour{white}{3}};

		\node [anchor=center,scale=\numscale] at (0.5,1.5) {\contour{white}{32}};
		\node [anchor=center,scale=\numscale] at (0.5,0.5) {\contour{white}{33}};
		\node [anchor=center,scale=\numscale] at (1.5,1.5) {\contour{white}{34}};
		\node [anchor=center,scale=\numscale] at (1.5,0.5) {\contour{white}{35}};

		\node [anchor=center,scale=\numscale] at (5.5,3.5) {\contour{white}{28}};
		\node [anchor=center,scale=\numscale] at (5.5,2.5) {\contour{white}{29}};
		\node [anchor=center,scale=\numscale] at (6.5,3.5) {\contour{white}{30}};
		\node [anchor=center,scale=\numscale] at (6.5,2.5) {\contour{white}{31}};

		\node [anchor=center,scale=\numscale] at (5.5,1.5) {\contour{white}{60}};
		\node [anchor=center,scale=\numscale] at (5.5,0.5) {\contour{white}{61}};
		\node [anchor=center,scale=\numscale] at (6.5,1.5) {\contour{white}{62}};
		\node [anchor=center,scale=\numscale] at (6.5,0.5) {\contour{white}{63}};

	\end{scope}
\end{tikzpicture}

    \caption{The pattern of inputs in a S-box}
    \label{fig:bitpattern}
\end{figure}
The goal of cryptography is to hide the contents of messages such that if a message between two parties is intercepted by a third, this third party will not easily be able to recover the message contents.
\end{document}
