\documentclass[a4paper,landscape,10pt]{article}
\usepackage[USenglish]{babel}
\usepackage{color,colortbl}
\usepackage{xcolor}
\usepackage{mathtools}
\usepackage{amsmath}
\usepackage{amsthm}
%\usepackage{fullpage}
\usepackage{parskip}
\usepackage{listings}
% http://tex.stackexchange.com/questions/1375/what-is-a-good-package-for-displaying-algorithms/1376#1376
%\usepackage{algorithmicx} % algorithms
\usepackage[noend]{algpseudocode}
\usepackage{algorithm} % algorithms
%\usepackage{algorithmic}        % algoritms
\usepackage{float}
\usepackage{relsize}    % for relative font sizing, e.g. \smaller
\usepackage{graphicx}
\usepackage{pgfplots}               % draw plots
%\pgfplotsset{compat=1.7}
\usepackage{pgfplotstable}          % fuck yeah! tables from tab-separated data
\usepackage{caption}
\usepackage{subcaption}
\usepackage[margin=0in]{geometry}
\usepackage{multicol}   % multiple columns if you want them
\usepackage{lmodern}    % typeface
%\usepackage{fancyhdr}
\usepackage{emptypage}  % can force pages to be empty
\usepackage{tikz}       % draw pictures
\usetikzlibrary{calc,decorations.markings,positioning} % draw bus-width lines
\usepackage{mathpazo}   % for the counter in the sudoku picture(?)
\usepackage{capt-of}    % for putting a figure in the margin
\usepackage{varwidth}   % for allowing variable-width contents of \set
\usepackage{multirow}
\usepackage{xfrac}
\usepackage{ulem}

\usepackage[T1]{fontenc} 
\usepackage{fontspec}
%\setmainfont{"Latin Modern Sans Serif"}
%\setmainfont{"Bitstream Vera Sans"}
\usepackage{csquotes}

\usepackage{ifthen}     % conditional in sudoku pictures

% for scaling up math
\newcommand*{\Scale}[2][4]{\scalebox{#1}{\ensuremath{#2}}}%

% use this for small caps when there's no smallcaps in the provided package or no bold-smallcaps
\renewcommand{\smaller}[1]{\relsize{-1}#1\relsize{+1}}
\newcommand{\cst}[1]{\relsize{-2}#1\relsize{+2}}
\renewcommand{\sc}[1]{\relsize{-2}{\uppercase{#1}}}
\newcommand{\func}[1]{#1}
% for empty node filling 
\newcommand{\isempty}[3]{
    \ifthenelse{\equal{#1}{}}{#2}{#3}
}
% correct sizing for curly braces
%\newcommand{\set}[1]{\begin{Bmatrix}#1\end{Bmatrix}}
\newcommand{\set}[1]{\left\{#1\right\}}

% for algorithms
\newcommand*\Let[2]{\State #1 $\gets$ #2}

% abs!
\DeclarePairedDelimiter\abs{\lvert}{\rvert}%
\DeclarePairedDelimiter\norm{\lVert}{\rVert}%
% Swap the definition of \abs* and \norm*, so that \abs
% and \norm resizes the size of the brackets, and the 
% starred version does not.
\makeatletter
\let\oldabs\abs
\def\abs{\@ifstar{\oldabs}{\oldabs*}}
\newcommand{\tuple}[1]{\left\langle #1\right\rangle}
%
\let\oldnorm\norm
\def\norm{\@ifstar{\oldnorm}{\oldnorm*}}
\makeatother

% remake text of FORALL
\algrenewcommand{\algorithmicforall}{\textbf{for each}}
\newcommand{\In}{~\textbf{in}~}
\newcommand{\Constrain}[1]{\State{\textbf{constrain}~#1}}

% trying to redefine the font used for math operators (names! e.g. max/linear)
\DeclareSymbolFont{sfoperators}{OT1}{cmss}{m}{n}
\DeclareSymbolFontAlphabet{\mathsf}{sfoperators}
\makeatletter
\def\operator@font{\mathgroup\symsfoperators}
\makeatother


\DeclareMathOperator{\parity}{parity}
\DeclareMathOperator{\weight}{weight}
\DeclareMathOperator{\alldifferent}{alldifferent}
\DeclareMathOperator{\lin}{linear}
\DeclareMathOperator{\score}{score}
\DeclareMathOperator{\assigned}{assigned}
\DeclareMathOperator{\disequal}{disequal}
\DeclareMathOperator{\fixed}{fixed}
\DeclareMathOperator{\free}{free}
\DeclareMathOperator{\funccount}{count}
\DeclareMathOperator{\funcsum}{sum}
\DeclareMathOperator{\xor}{xor}
\DeclareMathOperator{\xordist}{xordist}
\DeclareMathOperator{\channel}{channel}
\DeclareMathOperator{\nonlinear}{nonlinear}
\DeclareMathOperator{\weights}{weights}
\DeclareMathOperator{\propagate}{propagate}
\DeclareMathOperator{\valid}{valid}
\DeclareMathOperator{\store}{store}
\DeclareMathOperator{\sol}{sol}
\DeclareMathOperator{\var}{var}
\DeclareMathOperator{\select}{select}
\DeclareMathOperator{\dfe}{dfe}

% color style for "highlights" in sudoku
\usepackage[outline]{contour}
\definecolor{light-gray}{gray}{0.95}
\tikzset{sudokufill/.style={fill=light-gray}}
\tikzset{line/.style={<-,ultra thick}}
\tikzset{label/.style={scale=2.5,red,anchor=west,font=\bfseries}}
\tikzset{sel/.style={yellow!40!white}}
\tikzset{selcell/.style={orange!40!white}}
\contourlength{1.5pt}

\usepackage{xparse}
\newcommand{\vc}[1]{\topskip0pt\vspace*{\fill}{#1}\vspace*{\fill}}
\newcommand{\hone}[1]{\vc{\Huge{\textbf{#1}}}}
\newcommand{\fuck}[2]{\vc{{\Huge\textbf{#1}} \vspace{10pt} {#2}}}
\newcommand{\BigO}[1]{\ensuremath{\operatorname{O}\left(#1\right)}}
\newcommand{\FF}{\vphantom{gypRSTI$\left\{x12yO(1),\neq\ldots,12\right\}$}}

\begin{filecontents}{bitdecomphcactivity.tsv}
	n	numruns	numtimeouts	totaltime	averagetime	totalnodes	averagenodes
	8	25	0	814.308	0.033	463	18.000
	12	25	0	1918.891	0.077	1036	41.000
	16	25	0	4422.634	0.177	2207	88.000
	20	25	0	16483.753	0.659	7876	315.000
	24	25	0	53749.745	2.150	23421	936.000
\end{filecontents}

\begin{filecontents}{bitdecomphcdegree.tsv}
	n	numruns	numtimeouts	totaltime	averagetime	totalnodes	averagenodes
	8	25	0	2581.471	0.103	1747	69.000
	12	25	0	34500.936	1.380	19571	782.000
\end{filecontents}

\begin{filecontents}{bitdecomphcnone.tsv}
	n	numruns	numtimeouts	totaltime	averagetime	totalnodes	averagenodes
	8	25	0	932.675	0.037	447	17.000
	12	25	0	1824.132	0.073	925	37.000
	16	25	0	4395.142	0.176	2006	80.000
	20	25	0	11021.096	0.441	5121	204.000
	24	25	0	70151.340	2.806	27895	1115.000
	28	25	0	148851.017	5.954	71345	2853.000
\end{filecontents}

\begin{filecontents}{bitdecomphcrnd.tsv}
	n	numruns	numtimeouts	totaltime	averagetime	totalnodes	averagenodes
	8	25	0	1531.238	0.061	997	39.000
	12	25	0	8760.812	0.350	5368	214.000
	16	25	0	81304.611	3.252	44960	1798.000
\end{filecontents}

\begin{filecontents}{bithcactivity.tsv}
	n	numruns	numtimeouts	totaltime	averagetime	totalnodes	averagenodes
	8	25	0	430.957	0.017	466	18.000
	12	25	0	798.043	0.032	1034	41.000
	16	25	0	1591.649	0.064	2210	88.000
	20	25	0	5119.010	0.205	8003	320.000
	24	25	0	14866.096	0.595	23439	937.000
	28	25	1	109971.571	4.582	188025	7834.000
\end{filecontents}

\begin{filecontents}{bithcdegree.tsv}
	n	numruns	numtimeouts	totaltime	averagetime	totalnodes	averagenodes
	8	25	0	1032.844	0.041	1775	71.000
	12	25	0	11060.065	0.442	19665	786.000
	16	25	0	308676.506	12.347	547474	21898.000
\end{filecontents}

\begin{filecontents}{bithcnone.tsv}
	n	numruns	numtimeouts	totaltime	averagetime	totalnodes	averagenodes
	8	25	0	693.840	0.028	450	18.000
	12	25	0	1300.841	0.052	928	37.000
	16	25	0	2441.135	0.098	2020	80.000
	20	25	0	4730.319	0.189	5123	204.000
	24	25	0	28228.526	1.129	27948	1117.000
	28	25	0	48609.864	1.944	71523	2860.000
	32	25	1	120467.089	5.019	139460	5810.000
\end{filecontents}

\begin{filecontents}{bithcrnd.tsv}
	n	numruns	numtimeouts	totaltime	averagetime	totalnodes	averagenodes
	8	25	0	658.034	0.026	1008	40.000
	12	25	0	2955.336	0.118	5314	212.000
	16	25	0	24465.387	0.979	44558	1782.000
	20	25	0	270551.871	10.822	490770	19630.000
\end{filecontents}

\begin{filecontents}{bithcssevenactivity.tsv}
	n	numruns	numtimeouts	totaltime	averagetime	totalnodes	averagenodes
	8	25	0	138.887	0.006	468	18.000
	12	25	0	175.700	0.007	1036	41.000
	16	25	0	251.083	0.010	2206	88.000
	20	25	0	615.860	0.025	8349	333.000
	24	25	0	2593.939	0.104	38554	1542.000
	28	25	0	15803.285	0.632	304507	12180.000
	32	25	1	68986.430	2.874	1180315	49179.000
\end{filecontents}

\begin{filecontents}{bithcssevendegree.tsv}
	n	numruns	numtimeouts	totaltime	averagetime	totalnodes	averagenodes
	8	25	0	202.855	0.008	1782	71.000
	12	25	0	1196.935	0.048	19667	786.000
	16	25	0	31090.503	1.244	539230	21569.000
\end{filecontents}

\begin{filecontents}{bithcssevennone.tsv}
	n	numruns	numtimeouts	totaltime	averagetime	totalnodes	averagenodes
	8	25	0	144.036	0.006	452	18.000
	12	25	0	173.705	0.007	930	37.000
	16	25	0	255.139	0.010	2020	80.000
	20	25	0	412.919	0.017	5134	205.000
	24	25	0	1879.663	0.075	27990	1119.000
	28	25	0	3508.919	0.140	71555	2862.000
	32	25	0	10867.687	0.435	224745	8989.000
	36	25	0	50311.354	2.012	1031139	41245.000
	40	25	1	159737.712	6.656	3122800	130116.000
\end{filecontents}

\begin{filecontents}{bithcssevenrnd.tsv}
	n	numruns	numtimeouts	totaltime	averagetime	totalnodes	averagenodes
	8	25	0	165.240	0.007	1014	40.000
	12	25	0	392.728	0.016	5282	211.000
	16	25	0	2586.421	0.103	42080	1683.000
	20	25	0	28955.415	1.158	523154	20926.000
\end{filecontents}

\begin{filecontents}{bitinthcactivity.tsv}
	n	numruns	numtimeouts	totaltime	averagetime	totalnodes	averagenodes
	8	25	0	404.440	0.016	466	18.000
	12	25	0	801.932	0.032	1034	41.000
	16	25	0	1603.501	0.064	2210	88.000
	20	25	0	5112.378	0.204	8003	320.000
	24	25	0	14905.267	0.596	23439	937.000
	28	25	1	109753.147	4.573	188023	7834.000
\end{filecontents}

\begin{filecontents}{bitinthcdegree.tsv}
	n	numruns	numtimeouts	totaltime	averagetime	totalnodes	averagenodes
	8	25	0	1052.263	0.042	1771	70.000
	12	25	0	11161.616	0.446	19657	786.000
	16	25	0	310530.793	12.421	547474	21898.000
\end{filecontents}

\begin{filecontents}{bitinthcnone.tsv}
	n	numruns	numtimeouts	totaltime	averagetime	totalnodes	averagenodes
	8	25	0	406.211	0.016	450	18.000
	12	25	0	760.038	0.030	928	37.000
	16	25	0	1494.941	0.060	2020	80.000
	20	25	0	3366.927	0.135	5123	204.000
	24	25	0	17737.420	0.709	27948	1117.000
	28	25	0	34122.553	1.365	71523	2860.000
	32	25	0	98825.897	3.953	224457	8978.000
\end{filecontents}

\begin{filecontents}{bitinthcrnd.tsv}
	n	numruns	numtimeouts	totaltime	averagetime	totalnodes	averagenodes
	8	25	0	658.317	0.026	1008	40.000
	12	25	0	2983.003	0.119	5376	215.000
	16	25	0	24391.866	0.976	44558	1782.000
	20	25	0	270215.631	10.809	490770	19630.000
\end{filecontents}

\begin{filecontents}{boolinthcactivity.tsv}
	n	numruns	numtimeouts	totaltime	averagetime	totalnodes	averagenodes
	8	25	0	459.761	0.018	529	21.000
	12	25	0	1128.445	0.045	1401	56.000
	16	25	0	3077.541	0.123	3866	154.000
	20	25	0	15891.273	0.636	19245	769.000
	24	25	1	120199.265	5.008	142810	5950.000
\end{filecontents}

\begin{filecontents}{boolinthcdegree.tsv}
	n	numruns	numtimeouts	totaltime	averagetime	totalnodes	averagenodes
	8	25	0	532.089	0.021	527	21.000
	12	25	0	986.232	0.039	1099	43.000
	16	25	0	2413.298	0.097	2805	112.000
	20	25	0	4663.017	0.187	5468	218.000
	24	25	0	14007.293	0.560	16867	674.000
	28	25	0	49811.427	1.992	58257	2330.000
	32	25	0	112636.258	4.505	161046	6441.000
\end{filecontents}

\begin{filecontents}{boolinthcnone.tsv}
	n	numruns	numtimeouts	totaltime	averagetime	totalnodes	averagenodes
	8	25	0	500.245	0.020	527	21.000
	12	25	0	1060.982	0.042	1099	43.000
	16	25	0	2567.435	0.103	2805	112.000
	20	25	0	5008.443	0.200	5468	218.000
	24	25	0	14510.526	0.580	16867	674.000
	28	25	0	47635.543	1.905	58257	2330.000
	32	25	0	118704.614	4.748	161046	6441.000
\end{filecontents}

\begin{filecontents}{boolinthcrnd.tsv}
	n	numruns	numtimeouts	totaltime	averagetime	totalnodes	averagenodes
	8	25	0	440.015	0.018	539	21.000
	12	25	0	2644.344	0.106	3876	155.000
	16	25	0	20762.788	0.831	30708	1228.000
	20	25	0	369675.504	14.787	474317	18972.000
\end{filecontents}

\begin{filecontents}{setinthcactivity.tsv}
	n	numruns	numtimeouts	totaltime	averagetime	totalnodes	averagenodes
	8	25	0	1293.265	0.052	726	29.000
	12	25	0	2624.259	0.105	1489	59.000
	16	25	0	6622.631	0.265	3925	157.000
	20	25	0	25121.335	1.005	14527	581.000
	24	25	2	79408.244	3.453	40579	1764.000
	28	nan	nan	nan	nan	nan	nan
	32	nan	nan	nan	nan	nan	nan
\end{filecontents}

\begin{filecontents}{setinthcdegree.tsv}
	n	numruns	numtimeouts	totaltime	averagetime	totalnodes	averagenodes
	8	25	0	3420.642	0.137	2191	87.000
	12	25	0	26572.581	1.063	15897	635.000
	16	25	2	237933.901	10.345	125741	5467.000
\end{filecontents}

\begin{filecontents}{setinthcnone.tsv}
	n	numruns	numtimeouts	totaltime	averagetime	totalnodes	averagenodes
	8	25	0	1275.055	0.051	708	28.000
	12	25	0	2483.246	0.099	1374	54.000
	16	25	0	5028.034	0.201	2769	110.000
	20	25	0	11486.340	0.459	6131	245.000
	24	25	0	53833.631	2.153	29133	1165.000
	28	25	1	148863.564	6.203	88506	3687.000
\end{filecontents}

\begin{filecontents}{setinthcrnd.tsv}
	n	numruns	numtimeouts	totaltime	averagetime	totalnodes	averagenodes
	8	25	0	2205.645	0.088	1393	55.000
	12	25	0	14964.339	0.599	8849	353.000
	16	25	0	122909.307	4.916	67393	2695.000
\end{filecontents}

    \pgfplotstableread{bitdecomphcactivity.tsv}{\bitdecomphcactivity}
    \pgfplotstableread{bitdecomphcdegree.tsv}{\bitdecomphcdegree}
    \pgfplotstableread{bitdecomphcnone.tsv}{\bitdecomphcnone}
    \pgfplotstableread{bitdecomphcrnd.tsv}{\bitdecomphcrnd}

    \pgfplotstableread{bithcactivity.tsv}{\bithcactivity}
    \pgfplotstableread{bithcdegree.tsv}{\bithcdegree}
    \pgfplotstableread{bithcnone.tsv}{\bithcnone}
    \pgfplotstableread{bithcrnd.tsv}{\bithcrnd}

    \pgfplotstableread{bithcssevenactivity.tsv}{\bithcssevenactivity}
    \pgfplotstableread{bithcssevendegree.tsv}{\bithcssevendegree}
    \pgfplotstableread{bithcssevennone.tsv}{\bithcssevennone}
    \pgfplotstableread{bithcssevenrnd.tsv}{\bithcssevenrnd}

    \pgfplotstableread{bitinthcactivity.tsv}{\bitinthcactivity}
    \pgfplotstableread{bitinthcdegree.tsv}{\bitinthcdegree}
    \pgfplotstableread{bitinthcnone.tsv}{\bitinthcnone}
    \pgfplotstableread{bitinthcrnd.tsv}{\bitinthcrnd}

    \pgfplotstableread{boolinthcactivity.tsv}{\boolinthcactivity}
    \pgfplotstableread{boolinthcdegree.tsv}{\boolinthcdegree}
    \pgfplotstableread{boolinthcnone.tsv}{\boolinthcnone}
    \pgfplotstableread{boolinthcrnd.tsv}{\boolinthcrnd}

    \pgfplotstableread{setinthcactivity.tsv}{\setinthcactivity}
    \pgfplotstableread{setinthcdegree.tsv}{\setinthcdegree}
    \pgfplotstableread{setinthcnone.tsv}{\setinthcnone}
    \pgfplotstableread{setinthcrnd.tsv}{\setinthcrnd}

\begin{document}
\DeclareDocumentCommand\head{ m g g g g g g}{%
    \vc{%
        {\fontsize{36pt}{36pt}\selectfont\textbf{#1\FF}}%
        \IfNoValueF{#2}{\\\vspace{20pt}{\Huge #2\FF}}%
        \IfNoValueF{#3}{\\\vspace{20pt}{\Huge #3\FF}}%
        \IfNoValueF{#4}{\\\vspace{20pt}{\Huge #4\FF}}%
        \IfNoValueF{#5}{\\\vspace{20pt}{\Huge #5\FF}}%
        \IfNoValueF{#6}{\\\vspace{20pt}{\Huge #6\FF}}%
        \IfNoValueF{#7}{\\\vspace{20pt}{\Huge #7\FF}}%
    }%
}
\centering
\pagenumbering{gobble}
\pgfmathsetmacro{\scale}{0.75}

\fontfamily{lmss}\selectfont

\head{Implementation of bit-vector variables in a \\ constraint solver with an application to the \\ generation of cryptographic S-boxes}{\vspace{20pt}\fontsize{36pt}{36pt}\selectfont\color{gray}Kellen Dye}
\clearpage
% -----------------------------------------------------------------------------------------------------

\head{Prerequisites}{bit $\rightarrow$ "binary digit" $\rightarrow$ 0 or 1}{bit-vector $\rightarrow$ an array of bits, e.g. $1000110$}{bitwise operation $\rightarrow$ e.g. \sc{xor} ($\oplus$)}
\clearpage
% -----------------------------------------------------------------------------------------------------

\head{Constraint programming}
\clearpage
% -----------------------------------------------------------------------------------------------------

\head{Constraint programming}{Declarative $\rightarrow$ Describe the solution, not the process}
\clearpage
% -----------------------------------------------------------------------------------------------------


\head{Constraint programming}{Expresses relationships between variables}
\clearpage
% -----------------------------------------------------------------------------------------------------

\head{Constraint programming}{Relationships: e.g. $x \leq y$ or $x \neq y$}
\clearpage
% -----------------------------------------------------------------------------------------------------

\newcounter{row}
\newcounter{col}

\newcommand\setrow[9]{
  \setcounter{col}{1}
  \foreach \n in {#1, #2, #3, #4, #5, #6, #7, #8, #9} {
    \edef\x{\value{col} - 0.5}
    \edef\y{9.5 - \value{row}}
    \node[anchor=center,scale=2.0] at (\x, \y) {\n};
    \stepcounter{col}
  }
  \stepcounter{row}
}

\newsavebox{\sudoku}
\sbox{\sudoku}{%
\begin{tikzpicture}
    \begin{scope}
        \setcounter{row}{1}
        \setrow { }{2}{ }  {5}{ }{1}  { }{9}{ }
        \setrow {8}{ }{ }  {2}{ }{3}  { }{ }{6}
        \setrow { }{3}{ }  { }{6}{ }  { }{7}{ }

        \setrow { }{ }{1}  { }{ }{ }  {6}{ }{ }
        \setrow {5}{4}{ }  { }{ }{ }  { }{1}{9}
        \setrow { }{ }{2}  { }{ }{ }  {7}{ }{ }

        \setrow { }{9}{ }  { }{3}{ }  { }{8}{ }
        \setrow {2}{ }{ }  {8}{ }{4}  { }{ }{7}
        \setrow { }{1}{ }  {9}{ }{7}  { }{6}{ }

        \draw (0, 0) grid (9, 9);
        \draw[very thick, scale=3] (0, 0) grid (3, 3);
    \end{scope}
\end{tikzpicture}
}
\head{Sudoku}{Relationships between grid squares}{\usebox{\sudoku}}
\clearpage
% -----------------------------------------------------------------------------------------------------

\setcounter{row}{1}
\newsavebox{\sudokurow}
\sbox{\sudokurow}{%
\begin{tikzpicture}
    \begin{scope}
        \fill[sel] (0,1) rectangle ++(9,1);

        \setcounter{row}{1}
        \setrow { }{2}{ }  {5}{ }{1}  { }{9}{ }
        \setrow {8}{ }{ }  {2}{ }{3}  { }{ }{6}
        \setrow { }{3}{ }  { }{6}{ }  { }{7}{ }

        \setrow { }{ }{1}  { }{ }{ }  {6}{ }{ }
        \setrow {5}{4}{ }  { }{ }{ }  { }{1}{9}
        \setrow { }{ }{2}  { }{ }{ }  {7}{ }{ }

        \setrow { }{9}{ }  { }{3}{ }  { }{8}{ }
        \setrow {2}{ }{ }  {8}{ }{4}  { }{ }{7}
        \setrow { }{1}{ }  {9}{ }{7}  { }{6}{ }

        \draw (0, 0) grid (9, 9);
        \draw[very thick, scale=3] (0, 0) grid (3, 3);
    \end{scope}
\end{tikzpicture}
}
\head{Sudoku}{Row relationship: different values}{\usebox{\sudokurow}}

\clearpage
% -----------------------------------------------------------------------------------------------------

\setcounter{row}{1}
\newsavebox{\sudokucol}
\sbox{\sudokucol}{%
\begin{tikzpicture}
    \begin{scope}
        \fill[sel] (1,0) rectangle ++(1,9);

        \setcounter{row}{1}
        \setrow { }{2}{ }  {5}{ }{1}  { }{9}{ }
        \setrow {8}{ }{ }  {2}{ }{3}  { }{ }{6}
        \setrow { }{3}{ }  { }{6}{ }  { }{7}{ }

        \setrow { }{ }{1}  { }{ }{ }  {6}{ }{ }
        \setrow {5}{4}{ }  { }{ }{ }  { }{1}{9}
        \setrow { }{ }{2}  { }{ }{ }  {7}{ }{ }

        \setrow { }{9}{ }  { }{3}{ }  { }{8}{ }
        \setrow {2}{ }{ }  {8}{ }{4}  { }{ }{7}
        \setrow { }{1}{ }  {9}{ }{7}  { }{6}{ }

        \draw (0, 0) grid (9, 9);
        \draw[very thick, scale=3] (0, 0) grid (3, 3);
    \end{scope}
\end{tikzpicture}
}
\head{Sudoku}{Column relationship: different values}{\usebox{\sudokucol}}

\clearpage
% -----------------------------------------------------------------------------------------------------

\setcounter{row}{1}
\newsavebox{\sudokublock}
\sbox{\sudokublock}{%
\begin{tikzpicture}
    \begin{scope}
        \fill[sel] (0,0) rectangle ++(3,3);

        \setcounter{row}{1}
        \setrow { }{2}{ }  {5}{ }{1}  { }{9}{ }
        \setrow {8}{ }{ }  {2}{ }{3}  { }{ }{6}
        \setrow { }{3}{ }  { }{6}{ }  { }{7}{ }

        \setrow { }{ }{1}  { }{ }{ }  {6}{ }{ }
        \setrow {5}{4}{ }  { }{ }{ }  { }{1}{9}
        \setrow { }{ }{2}  { }{ }{ }  {7}{ }{ }

        \setrow { }{9}{ }  { }{3}{ }  { }{8}{ }
        \setrow {2}{ }{ }  {8}{ }{4}  { }{ }{7}
        \setrow { }{1}{ }  {9}{ }{7}  { }{6}{ }

        \draw (0, 0) grid (9, 9);
        \draw[very thick, scale=3] (0, 0) grid (3, 3);
    \end{scope}
\end{tikzpicture}
}
\head{Sudoku}{Block relationship: different values}{\usebox{\sudokublock}}
\clearpage
% -----------------------------------------------------------------------------------------------------

\setcounter{row}{1}
\newsavebox{\sudokuvar}
\sbox{\sudokuvar}{%
\begin{tikzpicture}
    \begin{scope}
        \fill[selcell] (1,1) rectangle ++(1,1);
        \fill[selcell] (5,1) rectangle ++(1,1);
        \setcounter{row}{1}
        \setrow { }{2}{ }  {5}{ }{1}  { }{9}{ }
        \setrow {8}{ }{ }  {2}{ }{3}  { }{ }{6}
        \setrow { }{3}{ }  { }{6}{ }  { }{7}{ }

        \setrow { }{ }{1}  { }{ }{ }  {6}{ }{ }
        \setrow {5}{4}{ }  { }{ }{ }  { }{1}{9}
        \setrow { }{ }{2}  { }{ }{ }  {7}{ }{ }

        \setrow { }{9}{ }  { }{3}{ }  { }{8}{ }
        \setrow {2}{ }{ }  {8}{ }{4}  { }{ }{7}
        \setrow { }{1}{ }  {9}{ }{7}  { }{6}{ }

        \draw (0, 0) grid (9, 9);
        \draw[very thick, scale=3] (0, 0) grid (3, 3);

        \coordinate (V) at (4.2,5.2);
        \node [scale=2.5,red,anchor=south west] at (V) {\contour{white}{variables}};
        \path [bend left,line,red] (1.75,1.75) edge (V);
        \path [bend left,line,red] (5.25,1.75) edge (V);
    \end{scope}
\end{tikzpicture}
}
\head{Sudoku}{Relationships between \sout{grid squares} variables}{\usebox{\sudokuvar}}
\clearpage
% -----------------------------------------------------------------------------------------------------

\head{Sudoku}{In constraints}{
    \begin{varwidth}{\textwidth}
        \begin{algorithmic}
             \For{$i \In \{1..9\}$}
                 \Constrain{$\alldifferent(row_i)$}
                 \Constrain{$\alldifferent(column_i)$}
                 \Constrain{$\alldifferent(block_i)$}
             \EndFor
        \end{algorithmic}
    \end{varwidth}
}
\clearpage
% -----------------------------------------------------------------------------------------------------

\head{Constraint solvers}{Reason on variables}\clearpage
% -----------------------------------------------------------------------------------------------------

\head{Constraint solvers}{Reason on \sout{variables} variable domains}\clearpage
% -----------------------------------------------------------------------------------------------------

\head{Variable domains}{e.g. "the values in $\set{1,\ldots, 9}$"}\clearpage
%\head{Variable domains}{potential values for this variable in a solution to the problem}\clearpage
% -----------------------------------------------------------------------------------------------------

\head{Propagators}{Enforce constraints on variable domains}\clearpage
% -----------------------------------------------------------------------------------------------------

\head{Propagators}{Identify impossible values and remove them: e.g. $\set{2,\ldots, 9}$}\clearpage
% -----------------------------------------------------------------------------------------------------

\head{Propagators}{Can cause other propagators to be executed}\clearpage
% -----------------------------------------------------------------------------------------------------

\head{Search}{When no more propagation can occur}\clearpage
% -----------------------------------------------------------------------------------------------------

\head{Search}{e.g. two search branches: $x < 5$ and $x \geq 5$}\clearpage
% -----------------------------------------------------------------------------------------------------

\head{Gecode (a constraint solver)}{Support for integers, Booleans, floats, sets}\clearpage
% -----------------------------------------------------------------------------------------------------

\head{Gecode (a constraint solver)}{\ldots\ but no bit-vectors}\clearpage
% -----------------------------------------------------------------------------------------------------

\head{Bit-vector variables}{Michel \& Van Hentenryck}\clearpage
% -----------------------------------------------------------------------------------------------------

\head{Bit-vector variables}{Potential $O(1)$ constraints!}\clearpage
% -----------------------------------------------------------------------------------------------------

\head{Bit-vector representation}{$\tuple{\mathit{lower}, \mathit{upper}}$}{Bits same in both lower and upper are assigned/fixed.}\clearpage
% -----------------------------------------------------------------------------------------------------

\head{Bit-vector variables}{Implemented in Gecode}\clearpage
% -----------------------------------------------------------------------------------------------------

\head{Bit-vector propagators}{Implemented some defined by Michel \& Van Hentenryck}\clearpage
% -----------------------------------------------------------------------------------------------------

\head{Bit-vector propagators}{New propagators for hamming weight, parity, disequality}\clearpage
% -----------------------------------------------------------------------------------------------------

\head{Substitution boxes}\clearpage
% -----------------------------------------------------------------------------------------------------

\head{Substitution boxes}{\Large\newcolumntype{g}{>{\columncolor{orange!40!white}}c}
\begin{tabular}{rr|llllllllglllllll}
\multicolumn{2}{c|}{}	& \multicolumn{16}{c}{Middle 4 bits} \\
\multicolumn{2}{c|}{\multirow{-2}{*}{$S_4$}} 
			&	0000	&	0001	&	0010	&	0011	&	0100	&	0101	&	0110	&	0111	&	1000	&	1001	&	1010	&	1011	&	1100	&	1101	&	1110	&	1111\\
\hline
	&	00	&	0111	&	1101	&	1110	&	0011	&	0000	&	0110	&	1001	&	1010	&	0001	&	0010	&	1000	&	0101	&	1011	&	1100	&	0100	&	1111\\
	&	01	&	1101	&	1000	&	1011	&	0101	&	0110	&	1111	&	0000	&	0011	&	0100	&	0111	&	0010	&	1100	&	0001	&	1010	&	1110	&	1001\\
\rowcolor{green!40!white}
\cellcolor{white}	&	10	&	1010	&	0110	&	1001	&	0000	&	1100	&	1011	&	0111	&	1101	&	\cellcolor{yellow}1111	&	0001	&	0011	&	1110	&	0101	&	0010	&	1000	&	0100\\
\multirow{-4}{*}{\shortstack{Outer\\bits}}
	&	11	&	0011	&	1111	&	0000	&	0110	&	1010	&	0001	&	1101	&	1000	&	1001	&	0100	&	0101	&	1011	&	1100	&	0111	&	0010	&	1110\\
\end{tabular}
}{$S_4(110000) = 1111$}\clearpage
% -----------------------------------------------------------------------------------------------------

\newsavebox{\feistel}
\pgfmathsetmacro{\figscale}{1}
\sbox{\feistel}{\begin{tikzpicture}[thick,scale=\figscale, every node/.style={transform shape}, font=\sffamily]
    \begin{scope}

        % drawing from the bottom up
        \pgfmathsetmacro{\h}{0}
        \pgfmathsetmacro{\I}{3}
        \pgfmathsetmacro{\O}{0.5}
        \pgfmathsetmacro{\H}{1}
        \pgfmathsetmacro{\Kscale}{1.5}
        \pgfmathsetmacro{\Koffset}{0.25}
        \pgfmathsetmacro{\Fscale}{2.0}
        \pgfmathsetmacro{\IOscale}{1.5}
        \pgfmathsetmacro{\R}{0.25}
        \pgfmathsetmacro{\D}{{2*\R}}
        \pgfmathsetmacro{\AR}{0.5}
        % total width = 2xpadding + fbox + 2xarrows + 2xradius (in arrows, xor)
        \pgfmathsetmacro{\W}{{1 + 1 + 1 + \R + \R}}
        \pgfmathsetmacro{\M}{{\W/2}}

        % output label
        \node [anchor=center] at (\W / 2,0) {ciphertext};
        \pgfmathsetmacro{\h}{\h + 0.25}

        % output box
        \draw (0,\h) rectangle ++(\W,\H);
        % dividing line
        \draw (\W / 2,\h) -- ++(0,\H);
        % output labels
        \node [anchor=center,scale=\IOscale] at (0.25 * \W,\h + 0.5) {$\text{R}_\text{n+1}$};
        \node [anchor=center,scale=\IOscale] at (3 * 0.25 * \W,\h + 0.5) {$\text{L}_\text{n+1}$};
        \pgfmathsetmacro{\h}{\h + \H}

        % arrows down to ouput
        \draw [<-] (\O,\h) -- ++(0,\H - \R);
        \draw [<-,red] (\I,\h) -- ++(0,\H + 0.5 + \R);
        \pgfmathsetmacro{\h}{\h + \H - \R}

        % last xor
        \draw (\O,\h + \R) circle [radius=\R];
        \draw (\O,\h) -- ++(0,2*\R);
        \draw (\O - \R, \h + \R) -- ++(2*\R,0);

        % arrow from F-box
        \draw [<-] (\O + \R, \h + \R) -- ++(\AR,0);
        % arrow into the F-box
        \draw [<-] (\I - \AR - \R, \h + \R) -- ++(\AR + \R,0);

        % f box
        \pgfmathsetmacro{\tmpy}{\h - \R}%(0.5 * \H)}
        \draw (\O + \R + \AR,\tmpy) rectangle ++(\H,\H);
        \pgfmathsetmacro{\tmpx}{\O + \R + \AR + (0.5 * \H)}
        \pgfmathsetmacro{\tmpy}{\tmpy + (0.5 * \H)}
        \node [anchor=center,scale=\Fscale] at (\tmpx,\tmpy) {F};

        % arrow for the key
        \pgfmathsetmacro{\ftop}{\tmpy + (0.5 * \H)}
        \draw [<-] (\M,\ftop) -- ++(0,0.5);

        % fbox and last xor on same horizontal
        \pgfmathsetmacro{\h}{\h + \D}

        % arrow into last xor
        \draw [<-] (\O,\h) -- ++(0,0.5);
        \pgfmathsetmacro{\h}{\h + 0.5}

        % key
        \node [anchor=center,scale=\Kscale] at (\M, \ftop + 0.5 + \Koffset) {$\text{K}_\text{n}$};

        % dotted lines into the Nth step
        \draw [densely dotted] (\O,\h) -- ++(0,0.5);
        \draw [densely dotted,red] (\I,\h) -- ++(0,0.5);
        \pgfmathsetmacro{\h}{\h + 0.5}
        
        % gap
        \pgfmathsetmacro{\h}{\h + 0.5}

        % dotted lines out of the 2nd step
        \draw [densely dotted] (\O,\h) -- ++(0,0.5);
        \draw [densely dotted,red] (\I,\h) -- ++(0,0.5);
        \pgfmathsetmacro{\h}{\h + 0.5}

        % connecting lines
        \draw (\O,\h) -- ++(\I - \O,0.5);
        \draw [red] (\I,\h) -- ++(\O - \I,0.5);
        \pgfmathsetmacro{\h}{\h + 0.5}

        %
        % begin second block
        %

        % just lines, not arrows, colors reversed
        \pgfmathsetmacro{\linesize}{0.5 + (\H / 2) - \R}
        \draw [red] (\O,\h) -- ++(0,\linesize);
        \draw (\I,\h) -- ++(0,2.5 + \R);
        \pgfmathsetmacro{\h}{\h + \linesize}

        % xor 1
        \draw (\O,\h + \R) circle [radius=\R];
        \draw (\O,\h) -- ++(0,2*\R);
        \draw (\O - \R, \h + \R) -- ++(2*\R,0);

        % arrow from F-box
        \draw [<-] (\O + \R, \h + \R) -- ++(\AR,0);
        % arrow into the F-box
        \draw [<-] (\I - \AR - \R, \h + \R) -- ++(\AR + \R,0);

        % f box
        \pgfmathsetmacro{\tmpy}{\h - \R}%(0.5 * \H)}
        \draw (\O + \R + \AR,\tmpy) rectangle ++(\H,\H);
        \pgfmathsetmacro{\tmpx}{\O + \R + \AR + (0.5 * \H)}
        \pgfmathsetmacro{\tmpy}{\tmpy + (0.5 * \H)}
        \node [anchor=center,scale=\Fscale] at (\tmpx,\tmpy) {F};

        % arrow for the key
        \pgfmathsetmacro{\ftop}{\tmpy + (0.5 * \H)}
        \draw [<-] (\M,\ftop) -- ++(0,0.5);

        % fbox and last xor on same horizontal
        \pgfmathsetmacro{\h}{\h + \D}

        % arrow into last xor
        \draw [<-,red] (\O,\h) -- ++(0,1.5);
        \pgfmathsetmacro{\h}{\h + 1.5}

        % key
        \node [anchor=center,scale=\Kscale] at (\M, \ftop + 0.5 + \Koffset) {$\text{K}_\text{1}$};

        % connecting lines
        \draw [red] (\O,\h) -- ++(\I - \O,0.5);
        \draw (\I,\h) -- ++(\O - \I,0.5);
        \pgfmathsetmacro{\h}{\h + 0.5}

        %
        % begin third (top) block
        %
        \draw (\O,\h) -- ++(0,\linesize);
        \draw [red] (\I,\h) -- ++(0,2.5 + \R);
        \pgfmathsetmacro{\h}{\h + \linesize}
        
        % xor 0
        \draw (\O,\h + \R) circle [radius=\R];
        \draw (\O,\h) -- ++(0,2*\R);
        \draw (\O - \R, \h + \R) -- ++(2*\R,0);

        % arrow from F-box
        \draw [<-] (\O + \R, \h + \R) -- ++(\AR,0);
        % arrow into the F-box
        \draw [<-] (\I - \AR - \R, \h + \R) -- ++(\AR + \R,0);

        % f box
        \pgfmathsetmacro{\tmpy}{\h - \R}%(0.5 * \H)}
        \draw (\O + \R + \AR,\tmpy) rectangle ++(\H,\H);
        \pgfmathsetmacro{\tmpx}{\O + \R + \AR + (0.5 * \H)}
        \pgfmathsetmacro{\tmpy}{\tmpy + (0.5 * \H)}
        \node [anchor=center,scale=\Fscale] at (\tmpx,\tmpy) {F};

        % arrow for the key
        \pgfmathsetmacro{\ftop}{\tmpy + (0.5 * \H)}
        \draw [<-] (\M,\ftop) -- ++(0,0.5);

        % fbox and last xor on same horizontal
        \pgfmathsetmacro{\h}{\h + \D}

        % arrow into last xor
        \draw [<-] (\O,\h) -- ++(0,1.5);
        \pgfmathsetmacro{\h}{\h + 1.5}

        % key
        \node [anchor=center,scale=\Kscale] at (\M, \ftop + 0.5 + \Koffset) {$\text{K}_\text{0}$};

        % output box
        \draw (0,\h) rectangle ++(\W,\H);
        % dividing line
        \draw (\W / 2,\h) -- ++(0,\H);
        % output labels
        \node [anchor=center,scale=\IOscale] at (0.25 * \W,\h + 0.5) {$\text{L}_\text{0}$};
        \node [anchor=center,scale=\IOscale] at (3 * 0.25 * \W,\h + 0.5) {$\text{R}_\text{0}$};
        \pgfmathsetmacro{\h}{\h + \H}

        % padding
        \pgfmathsetmacro{\h}{\h + 0.25}

        % input label
        \node [anchor=center] at (\W / 2,\h) {plaintext};
        \pgfmathsetmacro{\h}{\h + 0.25}

    \end{scope}
\end{tikzpicture}
}
\head{Substitution boxes}{Feistel network}{\usebox{\feistel}}\clearpage
% -----------------------------------------------------------------------------------------------------

\newsavebox{\desround}
\pgfmathsetmacro{\figscale}{2}
\sbox{\desround}{\begin{tikzpicture}[thick,scale=\figscale, every node/.style={transform shape}, font=\sffamily]
    \begin{scope}[decoration={markings,mark= at position 0.5 with {\node[font=\tiny] {/};}}]
        \pgfmathsetmacro{\h}{0}
        \pgfmathsetmacro{\sboxW}{1}
        \pgfmathsetmacro{\sboxP}{0.25}
	\pgfmathsetmacro{\sboxPH}{0.5 * \sboxP}
        %\pgfmathsetmacro{\W}{(8 * \sboxW) + (7 * \sboxP)}
        %\pgfmathsetmacro{\M}{\W / 2}

	\tikzset{sbox/.style={rectangle,minimum size=\sboxW,draw=black,anchor=west}}

        \pgfmathsetmacro{\h}{\h + 0.5}
        \pgfmathsetmacro{\h}{\h + 1}

	% s boxes
        \node (Sone) [sbox] at (0,\h){$S_1$};
        \node (Stwo) [sbox,right=\sboxP of Sone] {$S_2$};
        \node (Sthr) [sbox,right=\sboxP of Stwo] {$S_3$};
        \node (Sfou) [sbox,right=\sboxP of Sthr] {$S_4$};
        \node (Sfiv) [sbox,right=\sboxP of Sfou] {$S_5$};
        \node (Ssix) [sbox,right=\sboxP of Sfiv] {$S_6$};
        \node (Ssev) [sbox,right=\sboxP of Ssix] {$S_7$};
        \node (Seig) [sbox,right=\sboxP of Ssev] {$S_8$};

	% P box
	\coordinate (inP) at ($ (Sfou.south east) + (\sboxPH,-1) $);
        \node (P) [anchor=north,rectangle,draw=black,minimum size=\sboxW] at (inP) {P};
	\draw [<-] (P.north) -- ++(0,0.5);
        \draw [->,postaction={decorate}] (P.south) -- node[right=1pt] {\tiny 32} ++(0,-0.5);

	% out bus
	\coordinate (Fi) at ($(Sone.south) + (0,-0.5)$);
	\coordinate (La) at ($(Seig.south) + (0,-0.5)$);
	\draw (Fi) -- (La);

	% outgoing
	\draw [postaction={decorate}] (Sone.south) -- node[right=1pt] {\tiny 4} ++(0,-0.5);
	\draw [postaction={decorate}] (Stwo.south) -- node[right=1pt] {\tiny 4} ++(0,-0.5);
	\draw [postaction={decorate}] (Sthr.south) -- node[right=1pt] {\tiny 4} ++(0,-0.5);
	\draw [postaction={decorate}] (Sfou.south) -- node[right=1pt] {\tiny 4} ++(0,-0.5);
	\draw [postaction={decorate}] (Sfiv.south) -- node[right=1pt] {\tiny 4} ++(0,-0.5);
	\draw [postaction={decorate}] (Ssix.south) -- node[right=1pt] {\tiny 4} ++(0,-0.5);
	\draw [postaction={decorate}] (Ssev.south) -- node[right=1pt] {\tiny 4} ++(0,-0.5);
	\draw [postaction={decorate}] (Seig.south) -- node[right=1pt] {\tiny 4} ++(0,-0.5);
	
	% incoming
	\draw [<-,postaction={decorate}] (Sone.north) -- node[right=1pt] {\tiny 6} ++(0,0.5);
	\draw [<-,postaction={decorate}] (Stwo.north) -- node[right=1pt] {\tiny 6} ++(0,0.5);
	\draw [<-,postaction={decorate}] (Sthr.north) -- node[right=1pt] {\tiny 6} ++(0,0.5);
	\draw [<-,postaction={decorate}] (Sfou.north) -- node[right=1pt] {\tiny 6} ++(0,0.5);
	\draw [<-,postaction={decorate}] (Sfiv.north) -- node[right=1pt] {\tiny 6} ++(0,0.5);
	\draw [<-,postaction={decorate}] (Ssix.north) -- node[right=1pt] {\tiny 6} ++(0,0.5);
	\draw [<-,postaction={decorate}] (Ssev.north) -- node[right=1pt] {\tiny 6} ++(0,0.5);
	\draw [<-,postaction={decorate}] (Seig.north) -- node[right=1pt] {\tiny 6} ++(0,0.5);

	% in bus
	\coordinate (FiT) at ($(Sone.north) + (0,0.5)$);
	\coordinate (LaT) at ($(Seig.north) + (0,0.5)$);
	\draw (FiT) -- (LaT);

	\coordinate (xorOut) at ($ (Sfou.north east) + (\sboxPH,0.5) $);
	\draw (xorOut) -- ++(0,0.5);

        \pgfmathsetmacro{\R}{0.25} 
	\coordinate (xorC) at ($ (xorOut) + (0,0.5 + \R) $);
	\coordinate (xorL) at ($ (xorC) + (-\R,0) $);
	\coordinate (xorR) at ($ (xorC) + (+\R,0) $);

	\draw [red] (xorC) circle [radius=\R];
	\draw [red] (xorL) -- (xorR);
	\draw [red] ($ (xorC) + (0,-\R) $) -- ++(0,2*\R);

	\draw [<-] (xorL) -- ++(-1,0);
	\draw [<-] (xorR) -- ++(1,0);

	% E box
	\coordinate (outE) at ($ (xorL) + (-1,0.5) $);
        \node (E) [anchor=south,rectangle,draw=black,minimum size=\sboxW] at (outE) {E};
	\draw [postaction={decorate}] (E.south) -- node[right=1pt] {\tiny 48} ++(0,-0.5);

	% in to E
	\draw [<-,postaction={decorate}] (E.north) -- node[right=1pt] {\tiny 32} ++ (0,0.5);
	\node (blocktext) [anchor=south] at ($ (E.north) + (0,0.5) $) {Half\vphantom{y} block};

	\draw [postaction={decorate}] ($(xorR) + (1,0)$) -- node[right=1pt] {\tiny 48} ++ (0,1.5);
        \pgfmathsetmacro{\apart}{2 + (2 * \R)} 
	\node [anchor=south] at ($(xorR) + (1,1.5)$) {Subkey};

    \end{scope}
\end{tikzpicture}

}
\head{Substitution boxes}{Feistel function F}{\usebox{\desround}}\clearpage
% -----------------------------------------------------------------------------------------------------

\head{Constraint programming + S-boxes}{Suggested by Ramamoorthy et al.}\clearpage
% -----------------------------------------------------------------------------------------------------

\head{"Good" substitution boxes}{According to the DES design criteria}{Relationships between values in the S-box:}{ 
    \begin{varwidth}{\textwidth}
        \begin{itemize}
            \item $x \neq y$
            \item $\weight(S(x) \oplus S(y)) \geq 2$
            \item $\alldifferent(row_i)$
            \item $\score(S) \leq threshold$
            \item \ldots
        \end{itemize}
    \end{varwidth}
}\clearpage
% -----------------------------------------------------------------------------------------------------

\head{S-box constraints}{Implemented global propagators for DES design criteria S-2 and S-7}\clearpage
% -----------------------------------------------------------------------------------------------------
% add to H_C
% fix errors -> S-7

\head{Symmetries}{Reduce the search space}\clearpage
% -----------------------------------------------------------------------------------------------------

\head{Symmetries}{Ramamoorthy et al. describe several (rotation, "bit inversion")}\clearpage
% -----------------------------------------------------------------------------------------------------

\head{Symmetries}{New symmetries: reflective over both $x$ and $y$ axes}\clearpage
% -----------------------------------------------------------------------------------------------------

\head{Comparison}{Compare models based on set and Boolean variables}{with several models based on bit-vectors}\clearpage
% -----------------------------------------------------------------------------------------------------

\head{Models}{Boolean, global integer-based S-2}{set, global integer-based S-2}{bit-vector, decomposed S-2 and S-7}{bit-vector, global integer-based S-2, decomposed S-7}{bit-vector, global bit-vector-based S-2, decomposed S-7}{bit-vector, global bit-vector-based S-2, global bit-vector based S-7}\clearpage
% -----------------------------------------------------------------------------------------------------

\head{Results}{Bit-vector models are generally more effective}{With global propagators $\rightarrow$ much more effective}\clearpage
% -----------------------------------------------------------------------------------------------------

\head{Future work}{Implement additional propagators, integrate into Gecode core}{Alternate requirements/constraints for S-boxes}\clearpage
% -----------------------------------------------------------------------------------------------------

\head{Thanks!}\clearpage

\head{Graphs!}\clearpage

\pgfplotscreateplotcyclelist{kellencolors}{%
	teal,every mark/.append style={solid,fill=teal!60!black},mark=*\\%
	orange,every mark/.append style={solid,fill=orange!80!black},mark=square*\\%
	yellow!60!black,every mark/.append style={solid,fill=yellow!80!black},mark=triangle*\\%
	red!70!white,every mark/.append style={solid,fill=red!80!black},mark=diamond*\\%
	lime!80!black,every mark/.append style={solid,fill=yellow!80!black},mark=pentagon*\\%
    purple!60!black,every mark/.append style={solid,fill=red!80!black},mark=x\\%
}

\newsavebox{\rndbox}
\sbox{\rndbox}{%
            \begin{tikzpicture}[font=\sffamily]
                \begin{axis}[xlabel={\# unassigned variables}, 
                             xtick=data,
                             ytick=data,
                             ylabel={time (s)}, 
                             %title={RND}, 
                             width=0.6\textwidth, 
                             cycle list name=kellencolors,
                             legend style={legend pos=north west},
                             legend cell align=left]
                    \addplot +[line width=1.5pt] table[y=averagetime] from \setinthcrnd ;
                    \addlegendentry{set, integer S-2};
                    \addplot +[line width=1.5pt] table[y=averagetime] from \boolinthcrnd ;
                    \addlegendentry{boolean, integer S-2};
                    \addplot +[line width=1.5pt] table[y=averagetime] from \bitinthcrnd ;
                    \addlegendentry{bit-vector, integer S-2};
                    \addplot +[line width=1.5pt] table[y=averagetime] from \bitdecomphcrnd ;
                    \addlegendentry{bit-vector, decomposed S-2};
                    \addplot +[line width=1.5pt] table[y=averagetime] from \bithcrnd ;
                    \addlegendentry{bit-vector, S-2};
                    \addplot +[line width=1.5pt] table[y=averagetime] from \bithcssevenrnd ;
                    \addlegendentry{bit-vector, S-2, S-7};
                \end{axis}
            \end{tikzpicture}
        }
\head{RND}{\usebox{\rndbox}}\clearpage

\newsavebox{\degreebox}
\sbox{\degreebox}{%
            \begin{tikzpicture}[font=\sffamily]
                \begin{axis}[xlabel={\# unassigned variables}, 
                             xtick=data,
                             ytick=data,
                             ylabel={time (s)}, 
                             %title={DEGREE}, 
                             width=0.6\textwidth, 
                             cycle list name=kellencolors,
                             legend style={legend pos=north west},
                             legend cell align=left]
                    \addplot +[line width=1.5pt] table[y=averagetime] from \setinthcdegree ;
                    \addlegendentry{set, integer S-2};
                    \addplot +[line width=1.5pt] table[y=averagetime] from \boolinthcdegree ;
                    \addlegendentry{boolean, integer S-2};
                    \addplot +[line width=1.5pt] table[y=averagetime] from \bitinthcdegree ;
                    \addlegendentry{bit-vector, integer S-2};
                    \addplot +[line width=1.5pt] table[y=averagetime] from \bitdecomphcdegree ;
                    \addlegendentry{bit-vector, decomposed S-2};
                    \addplot +[line width=1.5pt] table[y=averagetime] from \bithcdegree ;
                    \addlegendentry{bit-vector, S-2};
                    \addplot +[line width=1.5pt] table[y=averagetime] from \bithcssevendegree ;
                    \addlegendentry{bit-vector, S-2, S-7};
                \end{axis}
            \end{tikzpicture}
        }
\head{DEGREE}{\usebox{\degreebox}}\clearpage

\newsavebox{\activitybox}
\sbox{\activitybox}{%
            \begin{tikzpicture}[font=\sffamily]
                \begin{axis}[xlabel={\# unassigned variables}, 
                             xtick=data,
                             ytick=data,
                             ylabel={time (s)}, 
                             %title={ACTIVITY}, 
                             width=0.6\textwidth, 
                             cycle list name=kellencolors,
                             legend style={legend pos=north west},
                             legend cell align=left]
                    \addplot +[line width=1.5pt] table[y=averagetime] from \setinthcactivity ;
                    \addlegendentry{set, integer S-2};
                    \addplot +[line width=1.5pt] table[y=averagetime] from \boolinthcactivity ;
                    \addlegendentry{boolean, integer S-2};
                    \addplot +[line width=1.5pt] table[y=averagetime] from \bitinthcactivity ;
                    \addlegendentry{bit-vector, integer S-2};
                    \addplot +[line width=1.5pt] table[y=averagetime] from \bitdecomphcactivity ;
                    \addlegendentry{bit-vector, decomposed S-2};
                    \addplot +[line width=1.5pt] table[y=averagetime] from \bithcactivity ;
                    \addlegendentry{bit-vector, S-2};
                    \addplot +[line width=1.5pt] table[y=averagetime] from \bithcssevenactivity ;
                    \addlegendentry{bit-vector, S-2, S-7};
                \end{axis}
            \end{tikzpicture}
        }
\head{ACTIVITY}{\usebox{\activitybox}}\clearpage

\newsavebox{\nonebox}
\sbox{\nonebox}{%
            \begin{tikzpicture}[font=\sffamily]
                \begin{axis}[xlabel={\# unassigned variables}, 
                             xtick=data,
                             ytick=data,
                             ylabel={time (s)}, 
                             %title={NONE}, 
                             width=0.6\textwidth, 
                             cycle list name=kellencolors,
                             legend style={legend pos=north west},
                             legend cell align=left]
                    \addplot +[line width=1.5pt] table[y=averagetime] from \setinthcnone ;
                    \addlegendentry{set, integer S-2};
                    \addplot +[line width=1.5pt] table[y=averagetime] from \boolinthcnone ;
                    \addlegendentry{boolean, integer S-2};
                    \addplot +[line width=1.5pt] table[y=averagetime] from \bitinthcnone ;
                    \addlegendentry{bit-vector, integer S-2};
                    \addplot +[line width=1.5pt] table[y=averagetime] from \bitdecomphcnone ;
                    \addlegendentry{bit-vector, decomposed S-2};
                    \addplot +[line width=1.5pt] table[y=averagetime] from \bithcnone ;
                    \addlegendentry{bit-vector, S-2};
                    \addplot +[line width=1.5pt] table[y=averagetime] from \bithcssevennone ;
                    \addlegendentry{bit-vector, S-2, S-7};
                \end{axis}
            \end{tikzpicture}
        }
\head{NONE}{\usebox{\nonebox}}\clearpage
                   
\end{document}
